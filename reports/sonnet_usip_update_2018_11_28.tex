\documentclass[]{article}
\usepackage{lmodern}
\usepackage{amssymb,amsmath}
\usepackage{ifxetex,ifluatex}
\usepackage{fixltx2e} % provides \textsubscript
\ifnum 0\ifxetex 1\fi\ifluatex 1\fi=0 % if pdftex
  \usepackage[T1]{fontenc}
  \usepackage[utf8]{inputenc}
\else % if luatex or xelatex
  \ifxetex
    \usepackage{mathspec}
  \else
    \usepackage{fontspec}
  \fi
  \defaultfontfeatures{Ligatures=TeX,Scale=MatchLowercase}
\fi
% use upquote if available, for straight quotes in verbatim environments
\IfFileExists{upquote.sty}{\usepackage{upquote}}{}
% use microtype if available
\IfFileExists{microtype.sty}{%
\usepackage{microtype}
\UseMicrotypeSet[protrusion]{basicmath} % disable protrusion for tt fonts
}{}
\usepackage[margin=1in]{geometry}
\usepackage{hyperref}
\hypersetup{unicode=true,
            pdftitle={Pakistani Candidate Asset Declaration Data - Progress and Update},
            pdfauthor={Luke Sonnet},
            pdfborder={0 0 0},
            breaklinks=true}
\urlstyle{same}  % don't use monospace font for urls
\usepackage{longtable,booktabs}
\usepackage{graphicx,grffile}
\makeatletter
\def\maxwidth{\ifdim\Gin@nat@width>\linewidth\linewidth\else\Gin@nat@width\fi}
\def\maxheight{\ifdim\Gin@nat@height>\textheight\textheight\else\Gin@nat@height\fi}
\makeatother
% Scale images if necessary, so that they will not overflow the page
% margins by default, and it is still possible to overwrite the defaults
% using explicit options in \includegraphics[width, height, ...]{}
\setkeys{Gin}{width=\maxwidth,height=\maxheight,keepaspectratio}
\IfFileExists{parskip.sty}{%
\usepackage{parskip}
}{% else
\setlength{\parindent}{0pt}
\setlength{\parskip}{6pt plus 2pt minus 1pt}
}
\setlength{\emergencystretch}{3em}  % prevent overfull lines
\providecommand{\tightlist}{%
  \setlength{\itemsep}{0pt}\setlength{\parskip}{0pt}}
\setcounter{secnumdepth}{0}
% Redefines (sub)paragraphs to behave more like sections
\ifx\paragraph\undefined\else
\let\oldparagraph\paragraph
\renewcommand{\paragraph}[1]{\oldparagraph{#1}\mbox{}}
\fi
\ifx\subparagraph\undefined\else
\let\oldsubparagraph\subparagraph
\renewcommand{\subparagraph}[1]{\oldsubparagraph{#1}\mbox{}}
\fi

%%% Use protect on footnotes to avoid problems with footnotes in titles
\let\rmarkdownfootnote\footnote%
\def\footnote{\protect\rmarkdownfootnote}

%%% Change title format to be more compact
\usepackage{titling}

% Create subtitle command for use in maketitle
\newcommand{\subtitle}[1]{
  \posttitle{
    \begin{center}\large#1\end{center}
    }
}

\setlength{\droptitle}{-2em}

  \title{Pakistani Candidate Asset Declaration Data - Progress and Update}
    \pretitle{\vspace{\droptitle}\centering\huge}
  \posttitle{\par}
    \author{Luke Sonnet}
    \preauthor{\centering\large\emph}
  \postauthor{\par}
      \predate{\centering\large\emph}
  \postdate{\par}
    \date{11/28/2018}


\begin{document}
\maketitle

This report outlines the goal of collecting data on Pakistani
candidates, current progress on data entry and cleaning, and what can be
expected in the final dataset in terms of the candidates included and
the covariates available.

\paragraph{Introduction}\label{introduction}

In the run up to the 2018 General Elections in Pakistan, all prospective
candidates must submit several forms to be approved as candidates in the
general elections. Two of these forms, a candidate affidavit form and a
statement of assets and liabilities were released by the Election
Commission of Pakistan (ECP) in late June, 2018, just under a month
before the elections.

These forms contain a rich set of data on self-declared candidate
wealth, tax payments, foreign holdings, outstanding criminal charges,
education, occupation, payments to and from political parties, and more.
Every candidate is supposed to submit these documents. The central
problem with these data is that they need not be released by the ECP,
and thus there are gaps in which candidates/constituencies are covered,
and that the data released are of questionable quality (e.g.~some
uploads are photographs of hand-written PDFs).

USIP, in a continued effort to release high-quality data around
elections in Central and South Asia, has funded an effort to enter this
data rigorously, and clean and prepare it so that it can be merged with
other important datasets, such as the election results and the official
tax payments reported by the tax authority of Pakistan.

\paragraph{Current status}\label{current-status}

The first wave of manual data entry by Research Solutions, a contracted
firm based in Lahore, Pakistan, with experience working with survey and
administrative economic and electoral data, was completed in September
of 2018. In the ensuing months, I have gone back and forth with Research
Solutions, asking them to reenter certain data, and to address certain
forms that may not have been entered. This process is ongoing, but is
approaching the final iteration.

In the meantime, I have begun cleaning and organizing the data,
preparing to merge it with the election results data and make it
amenable for analysis in a report and for future academic research.

\paragraph{Data coverage}\label{data-coverage}

Data will not be available for all candidates, even in the best case
scenario. Some of the PDFs that were released by the ECP are corrupt,
some are unreadable, and some files are missing altogether, as are some
constituencies.

In total, 17927 forms have been entered. There are only 11689 candidates
that contested the 2018 General Elections in open seats, implying that
many more forms were filed than candidates ended up contesting. This
could be due to candidates being disqualified or dropping out.

Of these 11689 candidates who contested the election, we have
successfully matched 10214, or 87.4 percent of them. This number may go
up as further efforts to plug gaps are being made. In particular,
Research Consultants is currently investigating 624 forms that were
released after the initial scrape of the ECP website that may shore up
some of these gaps.

Let's look at how this missingness breaks down by candidate rank,
constituency, and province. First, what percent of first place
candidates, second place candidates, and so on have we matched? The next
figure and table break it down, and shows that we generally recover
higher ranked candidates at higher rates (although it should be said
that after the fifth ranked candidate or so the number of total
candidates begins to fall as well). Note in the table the one ``NA''
candidate ran unopposed.

\begin{center}\includegraphics{sonnet_usip_update_2018_11_28_files/figure-latex/unnamed-chunk-2-1} \end{center}

\begin{longtable}[]{@{}lrrr@{}}
\toprule
Candidate rank & Matched candidates & Total candidates & Percent
matched\tabularnewline
\midrule
\endhead
First & 763 & 840 & 90.8\tabularnewline
Second & 749 & 840 & 89.2\tabularnewline
Third & 752 & 840 & 89.5\tabularnewline
Fourth & 754 & 840 & 89.8\tabularnewline
Fifth & 749 & 836 & 89.6\tabularnewline
Sixth+ & 6447 & 7492 & 86.1\tabularnewline
NA & 0 & 1 & 0.0\tabularnewline
\bottomrule
\end{longtable}

\newpage 

The next figure shows the number of constituencies (out of 840 that
contested this election), for which we have zero data, complete data, or
partial data coverage.

\begin{center}\includegraphics{sonnet_usip_update_2018_11_28_files/figure-latex/unnamed-chunk-3-1} \end{center}

Finally, we can see that much of the missingness comes from Balochistan
and Islamabad (where again the total number of candidates is low as
well). In Balochistan, some entire provinces are missing.

\begin{center}\includegraphics{sonnet_usip_update_2018_11_28_files/figure-latex/unnamed-chunk-4-1} \end{center}

\begin{longtable}[]{@{}lrrr@{}}
\toprule
Province & Matched candidates & Total candidates & Percent
matched\tabularnewline
\midrule
\endhead
Balochistan & 605 & 1233 & 49.1\tabularnewline
Islamabad & 37 & 69 & 53.6\tabularnewline
Khyber Paktunkhwa & 1771 & 1866 & 94.9\tabularnewline
Punjab & 5168 & 5510 & 93.8\tabularnewline
Sindh & 2633 & 3011 & 87.4\tabularnewline
\bottomrule
\end{longtable}

\newpage 

Lastly, there are both provincial and national assemblies. The national
assemblies are much higher profile, and we can see our coverage there is
also better:

\begin{longtable}[]{@{}lrrr@{}}
\toprule
Assembly & Matched candidates & Total candidates & Percent
matched\tabularnewline
\midrule
\endhead
National & 3036 & 3431 & 88.5\tabularnewline
Provincial & 7178 & 8258 & 86.9\tabularnewline
\bottomrule
\end{longtable}

\paragraph{Available covariates}\label{available-covariates}

After all of the merging is done, a considerable amount of cleaning and
estimating from the data will need to be done to create high quality
data. However, the covariates included will be at least these below
(although there may be missingness on some variables for some
observations):

\begin{itemize}
\tightlist
\item
  Identifiers

  \begin{itemize}
  \tightlist
  \item
    National tax number
  \item
    Computerized National Identity Card number
  \end{itemize}
\item
  Demographics

  \begin{itemize}
  \tightlist
  \item
    Education
  \item
    Occupation
  \item
    Number of children
  \item
    Location registered to vote
  \item
    Number and status of outstanding criminal cases
  \end{itemize}
\item
  Assets and liabilities

  \begin{itemize}
  \tightlist
  \item
    Total asset value this year
  \item
    Total asset value last year
  \item
    Total tax payments last three years
  \item
    Total liabilities
  \item
    Value of:

    \begin{itemize}
    \tightlist
    \item
      Domestic property
    \item
      Foreign property
    \item
      Capital holdings
    \item
      Vehicles
    \item
      Jewelry
    \item
      Cash deposits
    \item
      Other investments (by category)
    \end{itemize}
  \end{itemize}
\item
  Other data

  \begin{itemize}
  \tightlist
  \item
    Foreign passports
  \item
    Foreign trips number and cost
  \item
    Elected previously
  \end{itemize}
\end{itemize}


\end{document}
